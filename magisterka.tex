\documentclass[a4paper,12pt]{article}

\usepackage[utf8]{inputenc}
\usepackage{polski}
\usepackage[polish]{babel}
\usepackage{graphicx}

\providecommand{\imref}[1]{Rys. \ref{#1}} % referencja do obrazka

\begin{document}

\author{Jakub Kuźma}
\title{Pioneers - internetowa implementacja gry ,,Osadnicy z Catanu'' w oparciu o framework Ruby on Rails}
\date{\today}

\begin{titlepage}
\maketitle
\end{titlepage}

\section{Analiza zagadnienia}
\subsection{Osadnicy z Catanu}
Gra planszowa ,,Osadnicy z Catanu'' została stworzona przez
niemieckiego matematyka Klausa Teubera. Po raz pierwszy została ona
wydana w 1995 roku w Niemczech pod nazwą \emph{Die Siedler von
  Catan}. O ogromnej popularności gry na całym świecie może świadczyć
fakt, iż została ona przetłumaczona na ponad 20 języków, w tym język
polski. Na całym świecie organizowane są turnieje gry, a począwszy od
roku 2000 organizowane są coroczne mistrzostwa świata. Pierwsze
polskie wydanie gry przypada na rok 2005. Od tego czasu organizowane
są corocznie Oficjalne Mistrzostwa Polski w ,,Osadników z Catanu'',
które w 2009 roku odbędą się w Gliwicach.
\subsection{Elementy gry}
Pierwotna wersja gry jest przeznaczona dla 3-4 graczy. Słowo pierwotna
jest tutaj szczególnie warte podkreślenia, gdyż ze względu na swą
ogromną popularność, gra doczekała się bardzo dużej ilości dodatków i
modyfikacji zasad. W tym podrozdziale postaram się omówić wszystkie
elementy wchodzące w skład gry. Moim celem nie jest omawianie zasad
gry, a jedynie ukazanie najważniejszych jej aspektów, które musiały
zostać uwzględnione przy implementacji.

\subsubsection{Plansza}
Rozgrywka toczy się na planszy złożonej z 37 sześciokątów
oznaczających różne rodzaje terenu, oraz przypisanych do nich wartości
liczbowych. Poszczególne rodzaje pól mogą mieć także określony rodzaj
surowca, który może być z nich czerpany:

\begin{itemize}
\item las - drewno
\item pastwisko - wełna
\item pole uprawne - zboże
\item wzgórze - glina
\item góry - ruda żelaza
\item pustynia - brak surowca (początkowe położenie rozbójnika)
\item morze - brak surowca (może posiadać szlak handlowy)
\end{itemize}

Elementami planszy są również krawędzie i wierzchołki połączonych
sześciokątów. Na wierzchołkach (skrzyżowaniach) gracze mogą budować
osady i miasta. Przynoszą one dochody w postaci w.w. surowców z
sąsiadujących z nimi sześciokątów. Krawędzie natomiast przeznaczone są
pod budowę dróg, które łączą osady i miasta. Drogi nie przynoszą
żadnych dochodów, umożliwiają natomiast ekspansję terytorialną
(budowanie nowych osad).

Wszystkie surowce znajdujące się w posiadaniu poszczególnych graczy są
zakryte dla pozostałych uczestników gry. Zapamiętywanie surowców
otrzymanych przez pozostałych graczy jest bardzo istotnym elementem,
mającym duży wpływ na obraną strategię (m.in. na handel).

\subsubsection{Element losowy - kości do gry}
Gra ,,Osadnicy z Catanu'' jest grą umysłowo-losową. Głównym elementem
wprowadzającym do gry losowość, są dwie sześciościenne kości do
gry. Dla sumy dwóch rzutów kością prawdopodobieństwo uzyskania
poszczególnych wyników jest różne (zostało to przedstawione na
\imref{dice}). Sześciokąty na planszy (lasy, pastwiska, pola uprawne,
wzgórza i góry) mają przypisane wartości z przedziału 2-12, z
pominięciem liczby 7. Wszyscy gracze przed rozpoczęciem swojej tury
wykonują rzut kośćmi. Wynik rzutu oznacza, które sześciokąty przynoszą
w tej turze dochód. Surowce otrzymują tylko ci gracze, którzy są
posiadaczami osady lub miasta, które sąsiaduje z wylosowanym
sześciokątem. Przypisana wartość i rodzaj pola mają bezpośredni wpływ
na jego ,,atrakcyjność'' i co za tym idzie - na wybór strategii w
grze.

\begin{figure}[ht]
  \begin{center}
    \includegraphics[width=\linewidth]{dice.pdf}
  \end{center}
  \caption{Prawdopodobieństwo uzyskania wyniku dla rzutu dwiema
    szcześciościennymi kośćmi do gry}
  \label{dice}
\end{figure}

\subsubsection{Karty rozwoju}
Kolejnym czynnikiem wprowadzającym w niewielkim stopniu losowość są
karty rozwoju. Za określoną ilość kart surowców gracz może nabyć kartę
rozwoju, ciągnąc ją z wierzchu potasowanego i zakrytego stosu. W grze
występuje pięć rodzajów kart:

\begin{itemize}
\item karta rycerza
\item karty postępu (,,monopol'', ,,budowa dróg'' i ,,wynalazek'')
\item karta zwycięstwa
\end{itemize}

Karta rycerza stanowi element walki, element ten zostanie omówiony
później. Karty postępu w różny sposób przyspieszają rozwój
gracza. Ciekawym elementem jest jednak karta zwycięstwa, która stanowi
istotny element zaskoczenia w grze. W związku z tym, że po ,,zakupie''
karty rozwoju, jest ona zakryta dla pozostałych graczy, nie znają oni
rodzaju karty dopóki ta nie zostanie zagrana. Karty zwycięstwa
ujawniane są bezpośrednio przed końcem gry. W rezultacie gracz
posiadający np. trzy karty rozwoju, może potencjalnie posiadać trzy
dodatkowe punkty zwycięstwa, które umożliwią mu znacznie szybsze
zakończenie gry.

\subsubsection{Handel}
Podczas rozgrywki gracze mają możliwość wymiany posiadanych surowców z
innymi graczami lub z bankiem.

Handel z bankiem odbywa się po określonych ,,kursach
wymiany''. Początkowo niekorzystne kursy można zmieniać poprzez
budowanie portów (osad) przy szlakach handlowych. Pozwalają one nawet
na dwukrotne zwiększenie korzystności takich wymian.

Przy handlu z innymi uczestnikami nie ma żadnych narzuconych z góry
,,kursów wymiany'' (zależą one wyłącznie od graczy). Zawsze jednak
musi to być wymiana - zabronione jest pozbywanie się surowców.

\subsubsection{Walka}
Ostatnim istotnym elementem gry jest ,,walka''. Zagranie karty armii
oraz wyrzucenie siódemki daje możliwość przesunięcia pionka,
tzw. \emph{rozbójnika}. Pole zajmowane przez rozbójnika nie przynosi
graczom żadnego dochodu. Dodatkowo, po przesunięciu pionka, gracz ma
możliwość ,,obrabowania'' jednej z sąsiadujących z nim osad lub miast,
zabierając losowo wybraną kartę surowca od ich właściciela.

\subsubsection{Cel gry}
Celem gry jest zdobycie 10 punktów zwycięstwa, które są przyznawane
za:
\begin{itemize}
\item osadę - 1 punkt
\item miasto - 2 punkty
\item najdłuższą drogę - 2 punkty
\item największą armię - 2 punkty
\item kartę rozwoju ,,punkt zwycięstwa'' - 1 punkt
\end{itemize}



\end{document}
